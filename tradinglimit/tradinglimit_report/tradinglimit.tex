\documentclass[10pt,a4paper]{article}
\usepackage{amsmath,amssymb}
\usepackage{ragged2e}
\usepackage{graphicx}
\usepackage{float}

\title{\bf{Profindo Trading Limit untuk Margin}}
\author{Darmawan Putra Wijaya}
\date{\today}

\begin{document}
    \justifying
    \setlength{\parindent}{0pt}
    \maketitle
    \tableofcontents
    \bigskip

\section{Rumusan Masalah}
Sebelum nasabah melakukan transaksi short-selling pada akun margin,
Profindo perlu memastikan bahwa nasabah memiliki trading limit yang memadai.
Rumus trading limit yang digunakan Profindo adalah
\begin{equation}
    \text{Trading Limit}=2.85\times\text{Cash}+1.85\times\text{Total Collateral}.
\end{equation}
dengan kolateral dihitung berdasarkan
\begin{table}[H]
    \centering
    \begin{tabular}{|c|c|c|}
        \hline Kelas & Haircut & Capping \\ \hline
        A & 0-25\% & 12.5B \\ \hline
        B & 30-45\% & 6.5B \\ \hline
        C & 50-65\% & 2B \\ \hline
        D & 70-75\% & 2B \\ \hline
        E & 80-85\% & 2B \\ \hline
        F & 90-100\% & 0 \\ \hline
    \end{tabular}
\end{table}
Meskipun demikian, sistem Profindo (termasuk aplikasi Proclick) menggunakan parameter untuk menghitung trading limit sebagai berikut
\begin{figure}[H]
    \centering
    \includegraphics[width=0.9\textwidth]{parametersistemprofindo.png}
\end{figure}
Makna dari setiap parameter tidak diketahui dan 
diragukan terdapat perbedaan cara perhitungan trading limit dengan yang dikehendaki Profindo.
\\ \\
Pada Februari 2025, nilai Haircut KPEI saham ADMR naik menjadi 
\begin{equation}
    \text{Haircut KPEI}>65\%.
\end{equation}
Akibatnya, saham ADMR masuk ke dalam kelas saham D/E/F dan nilai kolateralnya diabaikan dalam perhitungan trading limit.
\\ \\
Atas komplain dari nasabah dengan argumen bahwa ADMR termasuk ke dalam indeks LQ45, 
maka Profindo menetapkan kebijakan bahwa Haircut Profindo saham ADMR ditetapkan menjadi 
\begin{equation}
    \text{Haircut Profindo}=65\%.
\end{equation}
Hal ini dapat menyebabkan Profindo perlu menanggung risiko yang lebih besar.
\\ \\
Maka akan diselidiki hal-hal berikut:
\begin{enumerate}
    \item Bagaimana cara sistem Profindo menghitung trading limit?
    \item Bagaimana interpretasi masing-masing parameter dalam sistem Profindo?
\end{enumerate}
\pagebreak

\section{Hipotesis}
Terdapat dugaan berdasarkan perbincangan dengan tim Risk Management dan IT bahwa sistem Profindo menghitung trading limit dengan menggunakan rumus
\begin{equation}
    TL_X=EBR_X\times(MC_X\times\text{Cash}+1.85\times\sum_{k=1}^{n}{\min(\text{Lot}_k\times100\times\text{Price}_k\times(1-\text{Haircut}_k),\text{Capping}_k)})
\end{equation}
dengan
\begin{align*}
    TL&=\text{Trading Limit}\\
    \text{Cash}&=\text{Jumlah kas pada portofolio}\\
    \text{Total Collateral}&=\text{Jumlah kolateral dengan jaminan saham pada portofolio}\\
    n&=\text{Jumlah kode saham berbeda pada portofolio}\\
    \text{Collateral}_k&=\text{Nilai pasar saham ke-k setelah dipotong haircut pada portofilio}\\
    \text{MVAHC}_k&=\text{Nilai pasar saham ke-k setelah dipotong haircut pada portofilio}\\
    \text{Capping}_k&=\text{Nilai capping saham ke-k pada portofolio}\\
    \text{MV}_k&=\text{Nilai pasar saham ke-k pada portofolio}\\
    \text{Lot}_k&=\text{Jumlah lot saham ke-k pada portofolio}\\
    \text{Haircut}_k&=\text{Nilai haircut Profindo untuk saham ke-k pada portofolio}
\end{align*}\

\section{Metodologi}
Akan dibuat puluhan hingga ratusan portofolio pada sistem Profindo dengan akun demo,
kemudian segala input dan output yang diterima dan dihasilkan sistem akan dicatat.
Proses pembuatan portofolio akan dibagi kedalam dua tahap
\begin{enumerate}
    \item Portofilio terdiri dari cash dan satu saham\\
    Tahap ini bertujuan untuk memastikan bahwa proses pemberlakuan Haircut, Capping, dan Multiplier 
    sesuai dengan yang terdapat pada hipotesis.
    \item Portofolio terdiri dari cash dan beberapa saham\\
    Tahap ini bertujuan untuk memastikan bahwa tidak ada variabel tambahan maupun variabel pengganggu lain.
\end{enumerate}
\pagebreak

\section{Hasil}

\pagebreak

\section{Kesimpulan}
\pagebreak

\end{document}